\documentclass[10pt]{article}

% Preamble

\usepackage{amsmath,amsfonts,amssymb}
\usepackage[mathscr]{euscript}
%\usepackage[mathcal]{euscript}
\usepackage{mathrsfs}
\usepackage{graphicx}
\usepackage{float}
\usepackage{bbm}
\usepackage{braket}
\usepackage{tikz-feynman}
\usepackage{simpler-wick}
\usepackage{cancel}
\usepackage{stackengine}

\newcommand{\bigzero}{\mbox{\normalfont\Large\bfseries 0}}

\title{Notes On Advanced Quantum Field Theory \\ The Theory of Elementary Interactions \\ A Course Given By Dr. Tobias Osborne}
\author{Transcribed by Dr. Alexander V. St. John}

% The Document

\begin{document}

\maketitle

\clearpage

\section*{Lecture 1: Introduction}
\label{sec: lec1}

\input{chapters/lec1.tex}

\clearpage

\section*{Lecture 2: Gaussian Path Integrals}
\label{sec: lec2}

\noindent Recall the propagator, or transition amplitude, for a nonrelativistic quantum system

\begin{equation}
U(q_i, q_f; T) = \left( \prod_j \int \mathcal{D} q^j (t) \int \mathcal{D} p^j (t) \right) e^{i \int^T_0 dt \, \mathcal{L} (q^j, \dot{q}^j)}.
\end{equation}

\noindent To work with this, we often discretize $q(t) \rightarrow q^j_k$

\begin{figure}[H]
	\centering
	\includegraphics[width=4in]{images/discretize.png}
\end{figure}

\begin{equation}
U(q_i, q_f; T) = \left( \prod_{j, k} \int dq^j_k \int \frac{dp^j_k}{2\pi} \right) e^{i \sum_k (\sum_j p_k^j (q_{k+1}^j - q_k^j ) - \epsilon H)}
\end{equation}

\noindent Evaluate these very many integrals to get an answer dependent on $\epsilon = \frac{T}{N}$, since we discretized, take the limit as $\epsilon \rightarrow 0$ and deal with any encountered infinities.

\subsection*{Key Example}

\noindent Consider the classical Hamiltonian

\begin{equation}
H = \frac{p^2}{2 m} + V(q).
\end{equation}

\noindent Calculate the transition amplitude (\textbf{Exercise})

\begin{align}
U(q_i, q_f; T) &= \left( \prod_{j, k} \int dq^j_k \int \frac{dp^j_k}{2\pi} \right) e^{i \sum_k (\sum_j p_k^j (q_{k+1}^j - q_k^j ) - \epsilon H)} \\
&=  \left( \prod_{k} \int dq_k \int \frac{dp_k}{2\pi} \right) e^{i \sum_k (p_k (q_{k+1} - q_k ) - \epsilon (\frac{p_k^2}{2m} + V(q)) )} \\
&= \left( \prod_k \int dq_k \right) \sqrt{\frac{-im}{2\pi \epsilon}} e^{i \sum_k \frac{m}{2\epsilon} (q_{k+1} - q_k)^2 - \epsilon V(\frac{q_{k+1} + q_k}{2})}.
\end{align}

\noindent We may also write this in the following notation, using the fact that the argument of the exponential is the discretized version of the action, now without the $p$-integral

\begin{equation}
\lim_{\epsilon \rightarrow 0} U(q_i, q_f; T) = \int \mathcal{D} q(t) e^{\mathcal{S}[q(t)]}
\end{equation}

\noindent Where the action is 

\begin{equation}
\mathcal{S}[q(t)] = \int^T_0 dt \,\, (\frac{m}{2} \sum_j (\dot{q}^j)^2 - V(q)).
\end{equation}

\noindent Note that if our system is a harmonic oscillator $V(q) = \frac{1}{2} m \omega q^2$, we can do the full integral.

\subsection*{Path Integrals for Scalar Fields}

\noindent Recall the classical scalar field with Lagrangian density and Hamiltonian

\begin{align}
\mathcal{L} &= \frac{1}{2} (\partial_\mu \phi)^2 - V(\phi) \\
H &= \int d^3 x \,\, (\frac{1}{2} \pi^2 + \frac{1}{2} (\nabla \phi)^2 + V(\phi) ).
\end{align}

\noindent The path integral prescription for quantum scalar fields gives the transition amplitude, by blind application of the above, we conjecture that

\begin{equation}
\bra{\phi_b} e^{-i \hat{H} T} \ket{\phi_a} = \left( \int \mathcal{D} \phi \int \mathcal{D} \pi \right) e^{i \int^T_0 d^4x \, (\pi \dot{\phi} - H(\phi)}
\end{equation}

\noindent Where the boundary terms are $\phi(t=0, x) = \phi_a (x)$ and $\phi(t=T, x) = \phi_b (x)$. \\

\noindent As explained above, to make sense of this quantity, we must discretize, evaluate, and take the continuum limit as $\epsilon \rightarrow \infty$. When we discretize, note that we only discretize space, as discretizing time in this way will cause trouble with the conjugate momenta. \\

\noindent The field operators are discretized over a "grid" of points $x_j$ each of width $\epsilon$, such that

\begin{equation}
\phi(t, x) \,\,\, \rightarrow \,\,\, \phi(t, x_j) \equiv q^j (t).
\end{equation}

\noindent Then discretize the integral by turning it into a sum over the grid

\begin{equation}
\int d^3 x  \,\,\, \rightarrow \,\,\, \epsilon^3 \Sigma_{j \in \mathbb{Z}^3} .
\end{equation}

\noindent Next the derivative can be discretized via a finite difference. Note that much better choices of a symmetric difference can be used, which are more computationally nice.

\begin{equation}
\nabla_\mu \phi(x) \,\,\, \rightarrow  \,\,\, \frac{(\phi(x_j + \epsilon_\mu) - \phi(x_j))}{|\epsilon_\mu|}.
\end{equation}

\noindent Where $\mu$ chooses one of four directions to calculate the derivative, and

\begin{equation}
\epsilon_{\mu} = \epsilon \{ \left(\begin{smallmatrix}1\\0\\0\\0\end{smallmatrix}\right), \left(\begin{smallmatrix}0\\1\\0\\0\end{smallmatrix}\right), \left(\begin{smallmatrix}0\\0\\1\\0\end{smallmatrix}\right), \left(\begin{smallmatrix}0\\0\\0\\1\end{smallmatrix}\right) \}
\end{equation}

\noindent And, lastly, the potential just becomes evaluated at each $x_j$

\begin{equation}
V(\phi(x)) \,\,\, \rightarrow \,\,\, V(\phi(x_j)).
\end{equation}

\noindent Then the Lagrangian is discretized to a sum over a bunch of terms (\textbf{Exercise}), but the only relevant term to the construction of the Hamiltonian is the time derivative of the field operator $\dot{\phi}$

\begin{equation}
L = \int d^3 x \, \mathcal{L} \rightarrow \epsilon^3 \sum_j \frac{1}{2} (\dot{\phi}_j)^2
\end{equation}

\noindent And the discretized conjugate momentum becomes

\begin{equation}
\pi^j = \frac{\partial L}{\partial \dot{q}^j} = \frac{\partial L}{\partial \dot{\phi}^j}  \,\,\, \rightarrow \,\,\, \epsilon^3 \dot{q}^j
\end{equation}

\noindent Finally, we have the discretized Hamiltonian, where we display the $\epsilon$ terms to show that if we did not add the $\epsilon^3$ term to the discretized Lagrangian, we would be stuck with an extra $\epsilon^{-3}$ on the discretized Hamiltonian

\begin{equation}
H = \epsilon^3 \sum_j \epsilon^{-3} \pi_j^2 + \frac{1}{2} \left( \frac{q_{j+\epsilon^\mu}-q_j}{\epsilon} \right)^2 + V(q).
\end{equation}

\noindent In summary, the discretization of the scalar field gives us a nonrelativistic lattice system such that the discretized Hamiltonian is the sum of a kinetic energy term and a potential energy term. The second step is to evaluate the (nonrelativistic) path integral, and the third step is to take the continuum limit as $\epsilon \rightarrow 0$, which will later be re-branded as renormalization. \\

\noindent The most important case of the scalar field is the quadratic potential, which corresponds to the Klein-Gordon field (e.g., discretizing Klein-Gordon theory yields the quadratic potential below), is

\begin{equation}
V(q) = \frac{1}{2} q^T \textbf{A} q.
\end{equation}

\subsection*{Gaussian Integrals}

\noindent Consider the following integral 

\begin{equation}
I = \int_{-\infty}^\infty dx \,\, e^{-x^2} = \sqrt{\pi}.
\end{equation}

\noindent Proof:

\begin{align}
I &= \int_{-\infty}^\infty dx \,\, e^{-x^2} \\
I^2 &= \left(\int_{-\infty}^\infty dx \,\, e^{-x^2} \right) \left( \int_{-\infty}^\infty dy \,\, e^{-y^2}  \right) \\
&= \int_0^\infty r dr \, \int_0^{2\pi} d\theta \, e^{-r^2} \\
I^2 &= 2 \pi \int_0^\infty \frac{d}{dr} \left( -\frac{1}{2} e^{-r^2} \right) dr = \pi
\end{align}

\noindent This is actually a special case of the more general form

\begin{align}
\int_{-\infty}^\infty dx \,\, e^{ -\frac{1}{2} a x^2 + b x } &= \sqrt{\frac{2\pi}{a}} e^{ \frac{b^2}{2a}  } \\
\int_{-\infty}^\infty dx \,\, e^{  i a x^2 + i b x } &= \sqrt{\frac{2\pi i}{a}} e^{ \frac{-i b^2}{2a}  } \\
\end{align}

\noindent We will extensively use the moments generated by the Gaussian integrand

\begin{align}
\langle x^n \rangle = \frac{\int_{-\infty}^\infty dx \,\, x^n e^{ -\frac{1}{2} a x^2}}{\int_{-\infty}^\infty dx \,\, e^{ -\frac{1}{2} a x^2}}.
\end{align}

\noindent Note that if $n$ is odd, then the moment is zero. So, rewrite the exponent as $2m$. Then (\textbf{Exercise})

\begin{equation}
\langle x^{2m} \rangle = \frac{(2m-1)!!}{a^m}.
\end{equation}

\noindent Note that the double factorial $(2m-1)!!$ represents the number of ways to join $2m$ points in pairs. -- "All science should in linear algebra or combinatorics." -- \\

\noindent Another closed form of this integral is in terms of derivatives

\begin{align}
\langle x^{2m} \rangle &=  \left( \frac{d}{db} \right)^{2m} \left(  \frac{\int_{-\infty}^\infty dx \,\, e^{ -\frac{1}{2} a x^2 + b x }}{\int_{-\infty}^\infty dx \,\, e^{ -\frac{1}{2} a x^2 }} \right) \big|_{b=0} \\
&= \left( \frac{d}{db} \right)^{2m} e^{\frac{b^2}{2a}} \big|_{b=0}.
\end{align}

\noindent To evaluate Gaussian integrals of many variables, where $x \in \mathbb{R}^n$, consider

\begin{equation}
I(\textbf{A}, B) = \int_{-\infty}^\infty dx_1 \dots \int_{-\infty}^\infty dx_n \,\, e^{-x^T \textbf{A} x + B^T x}
\end{equation}

\noindent Where $\textbf{A}$ is an $n \times n$ symmetric real matrix and $B$ is an $n \times 1$ real vector. Since $\textbf{A}$ is real, symmetric, it contains orthogonal $\textbf{O}$ and diagonal matrices $\textbf{D}$, such that $\textbf{O}^T \textbf{O} = \mathbb{I}$ and $\textbf{D}$ is diagonalized with the eignevalues of $\textbf{A}$.

\begin{equation}
\textbf{O}^T \textbf{D} \textbf{O} = \textbf{A}
\end{equation}

\noindent Assume that $B=0$ and define $y = \textbf{O} x$. Then

\begin{align}
I(\textbf{A}, B=0) &= \int_{-\infty}^\infty dy_1 \dots \int_{-\infty}^\infty dy_n \,\, e^{-y^T \textbf{D} y} \\
&= \prod_{j=1}^n \int_{-\infty}^\infty dy_j \,\, e^{-y_j^2 \lambda_j} \\
&= \prod_{j=1}^n \sqrt{\frac{\pi}{\lambda_j}} \\
I(\textbf{A}, B=0)&= \sqrt{\frac{\pi^n}{det(\textbf{A})}}.
\end{align}

\noindent The $B \ne 0$ case (\textbf{Exercise}) results in the following

\begin{equation}
I(\textbf{A},B) = \sqrt{\frac{\pi^n}{det(\textbf{A})}} e^{B^T \textbf{A}^{-1} B}.
\end{equation}

\clearpage

\section*{Lecture 3: Correlation Functions and Path Integrals}
\label{sec: lec3}

\input{chapters/lec3.tex}

\clearpage

\section*{Lecture 4: Functional Quantization of the Scalar Field}
\label{sec: lec4}

\input{chapters/lec4.tex}

\clearpage

\section*{Lecture 5: Functional Derivatives and Generating Functionals}
\label{sec: lec5}


\noindent Here we will finish the functional quantization of the scalar field. \\

\noindent Recall that we can compute time-ordered correlation functions for the quantum scalar field entirely in terms of classical quantities, which is equivalent to a sum over all diagrams,

\begin{equation}
\bra{\Omega} T[ \hat{\phi} (x_1) \dots \hat{\phi} (x_n) ] \ket{\Omega} = \lim_{T \rightarrow \infty(1-i\epsilon)} \frac{\int \mathcal{D} \phi \,\, \phi(x_1) \dots \phi(x_n) e^{i S[\phi(x_1), \dots , \phi (x_n)]}}{\int \mathcal{D} \phi \,\, e^{i S[\phi(x_1), \dots , \phi (x_n)]}}
\end{equation}

\noindent For example, the 2-point correlation function for the Klein-Gordon field is the Feynman propagator

\begin{equation}
\bra{\Omega} T[ \hat{\phi} (x_1) \hat{\phi} (x_2) ] \ket{\Omega} = \lim_{T \rightarrow \infty(1-i\epsilon)} \frac{\int \mathcal{D} \phi \,\, \phi(x_1) \phi(x_2) e^{i S}}{\int \mathcal{D} \phi \,\, e^{i S}} = D_F (x_1 - x_2)
\end{equation}

\noindent More elegantly, and analogous to multivariate Gaussian integrals, we found the Feynman propagator $D_F (x-y)$, which is the inverse operator of the Klein-Gordon operator $-\partial^2 - m^2$, to be similiar to the inverse of a matrix $\textbf{A}$, making the Klein-Gordon operator the matrix $\textbf{A}$. 

\begin{equation}
D_F (x_j - x_k) \sim [\textbf{A}^{-1}]_{jk} = \frac{\int dx_1 \dots dx_n \,\, x_j x_k e^{-\frac{1}{2} x^T \textbf{A} x} }{\int dx_1 \dots dx_n \,\, e^{-\frac{1}{2} x^T \textbf{A} x}}
\end{equation}

\noindent To compute these $n$-point correlation functions, or elements of this "inverse matrix", we used derivatives of the generating functional, which is what we generalize in this lecture. Recall the multivariate Gaussian generating functional

\begin{equation}
Z[J] = \int dx_1 \dots dx_n e^{ -\frac{1}{2} x^T \textbf{A} x - J^T x } = e^{ \frac{1}{2} J^T \textbf{A}^{-1} J }.
\end{equation}

\subsection*{Functional Derivatives}

\noindent The functional derivative is a tool from the calculus of variations that we now define by an example that is the continuum analog of the standard partial derivative

\begin{equation}
\frac{\delta}{\delta J(x)} J(y) = \delta^{(4)} (x-y).
\end{equation}

\noindent There is a \textit{one-to-one} mapping from the discrete representation to the continuous with correspondences

\begin{equation}
\begin{array}{ccc}
x \in \mathbb{R} & \rightarrow & j \in \mathbb{Z} \\
J(x) \in C(\mathbb{R}) & \rightarrow & J_j \in L_2(\mathbb{Z}) \\
\frac{\delta}{\delta J(x)} F[J(y)] & \rightarrow & \frac{\partial}{\partial J_j} F[J_1, J_2, \dots]
\end{array}
\end{equation}

\noindent Where $C(\mathbb{R})$ is a continuous function space and $L_2(\mathbb{Z})$ is ...

\subsubsection*{Example 1}

\begin{align*}
\frac{\delta}{\delta J(x)} e^{i \int d^4 y \, J(y) \phi (y)} &= i e^{i \int d^4 y \, J(y) \phi (y)} \frac{\delta}{\delta J(x)} \left( \int d^4 y \, J(y) \phi (y) \right) \\
&= i e^{i \int d^4 y \, J(y) \phi (y)} \int d^4 y \, \frac{\delta J(y)}{\delta J(x)} \phi(y) \\
&= i e^{i \int d^4 y \, J(y) \phi (y)} \int d^4 y \, \delta^{(4)} (x-y) \phi(y) \\
\frac{\delta}{\delta J(x)} e^{i \int d^4 y \, J(y) \phi (y)} &= i \phi (x) e^{i \int d^4 y \, J(y) \phi (y)}
\end{align*}

\subsubsection*{Example 2: Derivatives of Delta functions}

\begin{align*}
\frac{\delta}{\delta J(x)} \int d^4 y \, (\partial_\mu J(y) ) v^\mu (y) &= \frac{\delta}{\delta J(x)} \left( \text{boundary term}  - \int d^4 y \, J(y) \partial_\mu v^\mu (y) \right) \\
&= -\partial_\mu v^\mu (x)
\end{align*}

\noindent Note that the boundary term is almost always zero, except for topologically interesting theories.

\subsection*{The Generating Functional}

\noindent Define the generating functional as

\begin{equation}
Z[J] = \lim_{T \rightarrow \infty(1-i \epsilon)} \int \mathcal{D} \phi \,\, e^{i S + i J(x) \phi(x)}.
\end{equation}

\noindent This expression is obviously useful, since correlation functions are directly related to derivatives of $Z[J]$

\begin{equation}
\bra{\Omega} T [ \hat{\phi} (x) \hat{\phi} (y) ] \ket{\Omega} = \frac{- \frac{\delta}{\delta J(x)}  \frac{\delta}{\delta J(y)} Z[J] \big{|}_{J=0}}{Z[J] \big{|}_{J=0}}
\end{equation}

\noindent So, if you can compute the generating functional $Z[J]$, you have \textit{all} of the $n$-point correlation functiosn via derivatives for your field theory. \\

\noindent In free field theories, such as the Klein-Gordon field, the action is quadratic in the field operators, and the argument of the exponential is $Z[J]$ is

\begin{equation}
i(S_0 + J(x) \phi (x) = i \int d^4 x \,\, \left( \frac{1}{2} \phi (x) (-\partial^2 - m^2 + i \epsilon) \phi (x) + J(x) \phi(x) \right).
\end{equation}

\noindent To homogenize quadraticity, complete the square by introducing the shift (with Jacobian $=1$

\begin{equation}
\phi' (x) = \phi (x) - i \int d^4 y \,\, D_F (x-y) J (y).
\end{equation}

\noindent This is analogous to the positional shift $x' = x - \textbf{A}^{-1} J$, and works becuase the Feynman propagator is the inverse of the Klein-Gordon operator, such that 

\begin{equation}
(-\partial^2 - m^2 ) D_F (x-y) = i \delta^{(4)} (x-y)
\end{equation}

\noindent With the variable change, the exponential argument becomes

\begin{align*}
i(S_0 + J(x) \phi (x) = & i \int d^4 x \,\, \left( \frac{1}{2} \phi' (x) (-\partial^2 -m^2 + i \epsilon) \phi' (x) \right) \\
&- i \int d^4 x \int d^4 y \, \left( \frac{1}{2} J(x) (-i D_F (x-y)) J(y) \right).
\end{align*}

\noindent So, the generating functional is then

\begin{equation}
Z[J] = Z_0 e^{-\frac{1}{2} \int d^4 x d^4 y \, ( J(x) D_F (x-y) J (y))}
\end{equation}

\noindent Where the free field contribution, independent of $J$, is

\begin{equation}
Z_0 = \int \mathcal{D} \phi' \,\, e^{ i \int d^4 x \,\, \left( \frac{1}{2} \phi' (x) (-\partial^2 -m^2 + i \epsilon) \phi' (x) \right)}.
\end{equation}

\subsubsection*{Examples of Free Theory Correlations Functions}

\noindent \textbf{Example 1:} The 2-point correlation function, with the $Z_0$ cancelled out,

\begin{equation}
\bra{\Omega} T [ \hat{\phi} (x) \hat{\phi} (y) ] \ket{\Omega} = - \frac{\delta}{\delta J(x)}  \frac{\delta}{\delta J(y)} e^{-\frac{1}{2} \int d^4 x d^4 y \, ( J(x) D_F (x-y) J (y))} \big{|}_{J=0}.
\end{equation}

\noindent \textbf{Example:} The 4-point correlation function, with notation $\hat{\phi}_i = \hat{\phi} (x_i)$, $J_i = J (x_i)$, and $D_{xi} = D(x - x_i)$

\begin{align}
\bra{0} T[ \hat{\phi_1} \hat{\phi}_2 \hat{\phi}_3 \hat{\phi}_4 ] \ket{0} &= \frac{ \frac{\delta}{\delta J_1} \frac{\delta}{\delta J_2} \frac{\delta}{\delta J_3} \frac{\delta}{\delta J_4}  Z[J] \big{|}_{J=0} }{ Z[J=0] } \\
&= \frac{\delta}{\delta J_1} \frac{\delta}{\delta J_2} \frac{\delta}{\delta J_3} \left( - \int d^4 x' \,\, J_{x'} D_{x' 4} e^{-\frac{1}{2} \int d^4 x \int d^4 y J_x D_{xy} J_y} \right) \big{|}_{J=0}  \\
&= \frac{\delta}{\delta J_1} \frac{\delta}{\delta J_2} \left( -D_{34} + \int d^4 x' \int d^4 y' \, J_{x'} D_{x' 3} J_{y'} D_{y' 4} \right) \times e^{\dots} \big{|}_{J=0}   \\
&= \frac{\delta}{\delta J_1} \left( D_{34} \int d^4 x' \, J_{x'} D_{x' 2} + D_{24} \int d^4 y' \, J_{y'} D_{y' 3} + D_{23} \int d^4 z' \, J_{z'} D_{z' 4} + \mathcal{O} (J^2) \right) e^{\dots} \big{|}_{J=0}  \\
&= D_{34} D_{12} + D_{24} D_{13} + D_{23} D_{14}
\end{align}

\subsubsection*{Interacting Fields}

\noindent The time-ordered expectation value, which contains the generating functions by Taylor expansion, for the (classical) phi-fourth interacting theory is 

\begin{equation}
\bra{\Omega} T[ \phi_1 \dots \phi_n] \ket{\Omega} = \lim_{T \rightarrow \infty (1-i \epsilon )} \frac{ \int \mathcal{D} \, \phi e^{i(S_0 + S_{int})} \phi (x_1) \dots \phi (x_n) }{\int \mathcal{D} \phi \, e^{i(S_0 + S_{int})}}
\end{equation}

\noindent Where $S_{int} = -\frac{i \lambda}{4!} \int d^4 x \, \phi^4 (x)$. \\

\subsubsection*{Fermionic Fields}

\noindent For the (classical) \textit{fermionic field} $\hat{\psi}$, the 2-point correlation function, vacuum expectation value, is

\begin{equation}
\bra{\Omega} T[ \psi (x) \psi (y)] \ket{\Omega} = \frac{ \int \mathcal{D} \, \psi e^{iS} \psi (x) \psi (y) }{\int \mathcal{D} \psi \, e^{iS}}
\end{equation}

\noindent Rule number one for this expression (1) is to not think abou this operationally, and rule number two (2) is to think in analogy to complex numbers, which can provide a more clear representation and make things easier.  \\

\noindent The Fermi fields obey the relations

\begin{align}
\psi^2 (x) &= 0 \\
\psi (x) \psi (y) &= - \psi (y) \psi (x)
\end{align}

\subsection*{Vignette: Anticommuting Numbers (Grassman Numbers)}

\noindent Let $V$ be an $n$-dimensional vector space with basis $\theta_a \in V$, $a=1,\dots,n$. Thus, elements of the vector space $v \in V$ have the form 

\begin{equation}
v = \sum_{a=1}^n v_a \theta_a.
\end{equation}

\noindent To build a bigger vector space $\mathcal{G}(V)$ from $V$, we first endow $V$ with the product operation denoted by concatenation (e.g., $\theta_a \cdot \theta_b \cdot \theta_c = \theta_a \theta_b \theta_c$). \\

\noindent This gives us an infinite dimensional vector space with span

\begin{equation}
\mathcal{S}^{\infty} (V) = \text{span} \{ \theta_a, \theta_a \theta_b, \theta_a \theta_b \theta_c, \dots \}
\end{equation}

\noindent Now restrict the basis to obey the following suggestive relations

\begin{align}
\theta_a \theta_b &= - \theta_b \theta_a \\
\theta_a^2 &= 0.
\end{align}

\noindent Then the new vector space has dimension $\text{dim}(\mathcal{G} (V)) = 2^n$, and is the infinite dimensional span modulo the elements of the underlying vector space

\begin{equation}
\mathcal{G}(V) = \mathcal{S}^{\infty} (V) / v.
\end{equation}

\noindent (\textbf{Exercise}) Check that this structure is well-defined. Note that this is exactly the space of differential forms for a tangent space $V$ (also known as the space of "classical fermions"). \\

\noindent The basis of $\mathcal{G} (V)$ is now 

\begin{equation}
\{ 1, \, \theta_a, \, \theta_a \theta_b, \, \theta_a \theta_b \theta_c, \dots \}
\end{equation}

\noindent With $a=1,\dots,n$, followed by $1 \le a < b \le n$, $a < b < c$, etc. \\

\noindent Then a general element of the new vector space $f \in \mathcal{G} (V)$ is

\begin{align}
f = \alpha + \sum_{p=1}^n \,\, \sum_{1 \le j_1 < \dots < j_p \le n} \alpha_{j_1 j_2 \dots j_p} \, &\theta_{j_1} \theta_{j_2} \dots \theta_{j_p} \\
&, \text{where} \,\, \alpha_{j_1 j_2 \dots j_p} \in \mathbb{C}.
\end{align}

\clearpage

\section*{Lecture 6: Grassmann Numbers}
\label{sec: lec6}

\input{chapters/lec6.tex}

\clearpage

\section*{Lecture 7: Functional Quantization of the Dirac Field}
\label{sec: lec7}

\noindent We now employ Grassmann numbers/variables to build a path integral-like object that provides the $n$-point correlation functions for the Dirac (spinor) field. \\

\noindent Consider the Grassmann integral of the complex Grassmann variables $\theta$ and $\theta^*$, the Grassmann Gaussian generating function

\begin{equation}
Z[J] = \left( \prod_{j=1}^n \int d\theta^*_j d\theta_j \right) e^{- \sum_{j,k} \theta_j^* B_{jk} \theta_k + \sum_j (J_j^* \theta_j + \theta^*_j J_j) }
\end{equation}

\noindent Where the Grassmann variables and auxiliary fields $J$ and $J^*$ obey the anticommutation relations

\begin{align}
\{ \theta_j, \theta_k^* \} &= \{ \theta_j, \theta_k \} = \{\theta_j^*, \theta_k^* \} = 0 \\
\{J_j, \theta_k \} &= \{J_j^*, \theta_k \} = \{ J_j, J_k^* \} = 0.
\end{align}

\noindent Calculating these Gaussian integrals, the generating functional becomes

\begin{equation}
Z[J] = e^{- \sum_{j,k} J_j^* [B^{-1}]_{jk} J_k}.
\end{equation}

\noindent Since the matrix $B$ is unitary, such that $B^\dagger = B$, the generating functional $Z[J]$ is Hermitian. \\

\noindent Note that the generating functional takes a vector of Grassmann numbers, anticommuting objects, as input and yields an expression quadratic in the Grassmann numbers which evaluates to a real number, a commuting object, since observables correspond to Hermitian operators and real numbers as their eigenvalues; the expectation value must always be a real number, not a Grassmann number.

%
\noindent We now employ Grassmann numbers/variables to build a path integral-like object that provides the $n$-point correlation functions for the Dirac (spinor) field. \\

\noindent Consider the Grassmann integral of the complex Grassmann variables $\theta$ and $\theta^*$, the Grassmann Gaussian generating function

\begin{equation}
Z[J] = \left( \prod_{j=1}^n \int d\theta^*_j d\theta_j \right) e^{- \sum_{j,k} \theta_j^* B_{jk} \theta_k + \sum_j (J_j^* \theta_j + \theta^*_j J_j) }
\end{equation}

\noindent Where the Grassmann variables and auxiliary fields $J$ and $J^*$ obey the anticommutation relations

\begin{align}
\{ \theta_j, \theta_k^* \} &= \{ \theta_j, \theta_k \} = \{\theta_j^*, \theta_k^* \} = 0 \\
\{J_j, \theta_k \} &= \{J_j^*, \theta_k \} = \{ J_j, J_k^* \} = 0.
\end{align}

\noindent Calculating these Gaussian integrals, the generating functional becomes

\begin{equation}
Z[J] = e^{- \sum_{j,k} J_j^* [B^{-1}]_{jk} J_k}.
\end{equation}

\noindent Since the matrix $B$ is unitary, such that $B^\dagger = B$, the generating functional $Z[J]$ is Hermitian. \\

\noindent Note that the generating functional takes a vector of Grassmann numbers, anticommuting objects, as input and yields an expression quadratic in the Grassmann numbers which evaluates to a real number, a commuting object, since observables correspond to Hermitian operators and real numbers as their eigenvalues; the expectation value must always be a real number, not a Grassmann number. \\

\noindent Recall the Dirac spinor field which is what we mean to be classical fermions. The classical fermion is represented by the 4-component spacetime vector

\begin{equation}
\psi (x) \rightarrow M(\Lambda) \psi (\Lambda^{-1} x)
\end{equation}

\noindent With the representation of the Lorentz group

\begin{equation}
M = e^{-\frac{i}{2} \omega_{\mu\nu} S^{\mu \nu}}.
\end{equation}

\noindent Where $S^{\mu\nu} = \frac{i}{4} [\gamma^\mu, \gamma^\nu]$ and $\{ \gamma^\mu, \gamma^\nu \} = 2 \eta^{\mu\nu}$. An object that transforms according to the transformation law with this representation we call a Dirac spinor. The representation, and thus the generators, of the Poincar\'e group easily follows. \\

\noindent The Hamiltonian, or generator of time translations, that follows from this spinor object solves the Dirac equation $(i\partial_\mu \gamma^\mu - m) \psi = 0$. Recall that we defined the conjugate-like object $\bar{\psi} = \psi^\dagger \gamma^0$ to induce Lorentz invariance for the Lagrangian density $\mathcal{L} = \bar{\psi} (-i\slashed{\partial} - m) \psi$, where the slash notation denotes $\slashed{A} = A_\mu \gamma^\mu$. \\

\noindent Thus far, we've been thinking about $\psi = (\psi_1, \psi_2, \psi_3, \psi_4)$ as the classical spinor-valued Dirac field, but it is really the single-particle component that is needed to build the classical Dirac field with anticommuting objects, since we guess to employ the anticommutation relation of quantum Dirac field operators 

\begin{equation}
\{ \hat{\psi} (x), \hat{\psi}^\dagger (y) \} = \delta^{(4)} (x-y).
\end{equation}

\noindent So, to build a quantum theory, recall that we, in the case of creation and annihilation operators, for example,

\begin{enumerate}
\item Pick a classical single-particle theory
\item Put hats on the field operators to quantize them
\item Make an algebra for the quantized field operators to obey
\item Find representations of that algebra.
\end{enumerate}

\noindent Alternatively, we can find some classical Dirac field that we quantize via the path integral, and use the path integral as a tool to guess the quantum theory. \\

\noindent \textbf{Guess 1 (Wrong):} \\

\noindent Let the classical field $\psi (x)$ consist of real numbers. Then the corresponding path integral $\int \mathcal{D} \psi \mathcal{D} \bar{\psi} \, e^{iS}$ will not yield the $n$-point correlation functions for the quantum Dirac (fermionic) field, but will yield the $n$-point correlation functions for the bosonic field. \\

\noindent \textbf{Guess 2 (Correct):} \\

\noindent Let the classical field $\psi (x)$ consist of Grassmann numbers, a Grassmann-valued field as the classical Dirac field. Then the path integral will yield the quantum Dirac field. \\

\noindent A Grassmann-valued field (4-vectors) can be understood via sheaf theory and ringed spaces of Grassmann numbers on manifolds. \\

\noindent An alternative way to make sense of a Grassman-valued field is by discretization to a lattice of spacing $\epsilon$, compactified to a torus. Consider a map from spacetime to the 4-dimensional torus

\begin{equation}
\mathbb{R}^{1,3} \rightarrow (\mathbb{Z} / N \mathbb{Z})^{\otimes 4}
\end{equation}

\noindent Consider the $(0+1)$-dimensional case, mapping continuous spacetime coordinates to discrete coordinates: $x_j \rightarrow \epsilon j$, where $j \in \mathbb{Z}/N\mathbb{Z}$, and discretize the Grassmann numbers to the lattice to define the classical Dirac field

\begin{align}
\psi_j &\equiv \psi (x_j) \equiv \psi (\epsilon j) \\
\psi_j^\dagger &\equiv \psi^\dagger (x_j) \equiv  \psi^\dagger (\epsilon j).
\end{align}

\noindent So, the (discrete) classical Dirac field is a list of $8 \cdot N^4$ Grassmann numbers, since $N^4$ is the number of lattice sites and each of the two field ``operator'' contains 4 components. \\

\noindent Note that if we work in momentum space, the Fourier coefficients will be made to be Grassmann numbers. \\

\noindent The continuous classical Dirac field comes from the limit of the lattice spacing vanishing $\epsilon \rightarrow 0$, the number of sites tending to infinity $N \rightarrow \infty$, and the size of the torus tending to infinity $L = N \epsilon \rightarrow \infty$. \\

\noindent Now, to build the quantum theory corresponding to these classical objects via the path integral formalism, we require an action, beginning with the discretization of the Lagrangian density $\mathcal{L} = \bar{\psi} ( i \slashed{\partial} - m ) \psi$. For the Dirac field, the discretized Lagrangian density is

\begin{equation}
\mathcal{L} (\psi_j, \bar{\psi}_j ) = \sum_{j \in ( \mathbb{Z}/N\mathbb{Z})^{\otimes 4}} i \bar{\psi}_j \left( \gamma^\mu \left( \frac{\psi_{j + \hat{\mu}} - \psi_j}{\epsilon} \right) \right) - m \bar{\psi}_j \psi_j.
\end{equation}

\noindent Where $\psi_j$ and $\bar{\psi}_j$ are $4D$ spinors of Grassmann numbers, we've employed the forward-difference to represent the partial derivative, and $\hat{\mu}$ is a unit vector in the $\mu^{th}$ direction. \\

\noindent With the Lagrangian density, we can calculate the action $S = i \int^T_{-T} dt \, \mathcal{L}(\psi_j, \bar{\psi}_j )$ and the $n$-point correlation functions for the Grassmann-valued quantum field operators. \\

\noindent For example, for the Grassmann variables, define the 2-point correlation function

\begin{equation}
\bra{0} \mathcal{T} [ \hat{\psi} (x), \hat{\bar{\psi}} (y) ] \ket{0} \equiv \lim_{T \rightarrow \infty (1-i \epsilon)} \frac{\int \mathcal{D} \bar{\psi} \mathcal{D} \psi \,\, \psi(x) \bar{\psi} (y) e^{iS}}{\int \mathcal{D} \bar{\psi} \mathcal{D} \psi \,\, e^{iS}}
\end{equation}

\noindent To calculate the 2-point correlation function, we continue to follow the prescription

\begin{enumerate}
\item Discretize the field and the action
\item Evaluate the path integral
\item Take the continuous limit
\end{enumerate}

\noindent Evaluate the path integral to find that the 2-point function is

\begin{equation}
\bra{0} \mathcal{T} [ \hat{\psi} (x), \hat{\bar{\psi}} (y) ] \ket{0} = S_F (x-y) = \int \frac{d^4 k}{(2 \pi)^4} \, \frac{i e^{-i k \cdot (x-y)}}{\slashed{k} - m +i \epsilon}
\end{equation}

\noindent Side note (topic of ongiong research): Fermion doubling is a topological artifact of incorrectly placing fermions on a lattice and taking the continuous limit. Extra fermions, called doublers, appear in the calculation, as the dispersion relation $\omega (k)$ becomes nonlinear and crosses the $k$-axis more than once. The expected dispersion relation is linear $\omega (k) = a k$, $a > 0$. In discretization, we must accept this effect and learn how to work around it. This is done for conveience, since without discretization, evaluating the 2-point function requires many more tricks.

\subsection*{Generating Functional for the Dirac Field}

\noindent Define the generating functional for the Dirac field in terms of two independent Grassmann-valued functions 

\begin{equation}
Z[J(x),\bar{J}(y)] \equiv \int \mathcal{D} \bar{\psi} \mathcal{D} \psi \,\, e^{i \int d^4 x \,\, ( \bar{\psi} (i \slashed{\partial} - m ) \psi + \bar{J} \psi + \bar{\psi} J )} .
\end{equation}

\noindent Where $J$ and $\bar{J}$ are Grassmann-valued (auxiliary) source fields that will be set to zero after differentiation. Calculating the generating functional will yield all $n$-point functions via functional derivatives, made possible by the employment of Grassmann numbers and functional quantization versus canonical quantization. By completing the square and simplifying the expression for the generating functional we get (\textbf{Exercise})

\begin{equation}
Z[J,\bar{J}] = Z_0 e^{-\int d^4 x d^4 y \,\, \bar{J} (x) S_F (x-y) J(y) }
\end{equation}

\noindent Where $Z_0 = Z [ J=0, \bar{J} = 0]$. Recall that for Grassmann numbers, the rules of differentiation include sign-switching and go like

\begin{equation}
\frac{d}{d\eta} \theta \eta = - \theta \frac{d}{d \eta} \eta = - \theta
\end{equation}

\noindent So the $n$-point correlation function is then

\begin{equation}
\bra{0} \mathcal{T} [ \psi^{(\alpha_1)} (x_1) \dots \psi^{(\alpha_n)} (x_n) ] \ket{0} = Z_0^{-1} \left( i (-1)^{\alpha_1 + 1} \frac{\delta}{\delta J^{\alpha_1}} \right) \dots \left( i (-1)^{\alpha_n + 1} \frac{\delta}{\delta J^{\alpha_n}} \right) Z[J,\bar{J}]
\end{equation}

\noindent Where $\psi^{(\alpha)} (x) = \psi (x)$ for $\alpha=0$ and $\psi^{(\alpha)} (x) = \bar{\psi} (x)$ for $\alpha=1$. \\

\noindent Check (\textbf{Exercise}) that the quantum 2-point correlation function comes out to be the expected

\begin{equation}
\bra{0} \mathcal{T} [ \hat{\psi} (x) \hat{\bar{\psi}} (y) ] \ket{0} = S_F (x-y).
\end{equation}

\subsection*{Interactions of Fermions and Bosons}

\noindent The path integral is a great tool for guessing Feynman rules as well, since we can expand in a Taylor series and recognize patterns that represent certain symmetries and diagrams. WIthout needing to introduce too much gauge theory, we introduce \textit{massive quantum electrodynamics (QED)}, a quantum field theory that models the interaction of fermions and (massive) bosons. We expect the photon (boson) mass to be zero, but consider it massive for now, and note that the upper bounds on the mass of the photon have been calculated to be nonzero (~$10^{-20}$). \\

\noindent Consider the Lagrangian density

\begin{equation}
\mathcal{L} = \bar{\psi} ( i \gamma^\mu (\partial_\mu - i e A_\mu) - m_f ) \psi - \frac{1}{4} F^{\mu\nu}F_{\mu\nu} + \frac{1}{2} m_b^2 A_\mu A^\mu.
\end{equation}

\noindent There are six fields represented here: the fermion fields $\psi$ and $\bar{\psi}$, the 4 boson fields $A_\mu$, and the tensor $F^{\mu\nu} = \partial^\mu A^\nu - \partial^\nu A^\mu$. \\

\noindent Following the path integral quantization, the classical 2-point correlation function is

\begin{equation}
\bra{0} \mathcal{T} [ \psi (x) \bar{\psi} (y) ] \ket{0} = \lim_{T \rightarrow \infty (1-i \epsilon)} \frac{\int \mathcal{D} \psi \mathcal{D} \bar{\psi} \mathcal{D} A \,\, \psi (x) \bar{\psi} (y) e^{iS}}{\int \mathcal{D} \psi \mathcal{D} \bar{\psi} \mathcal{D} A \,\, e^{iS}}.
\end{equation}

\noindent Write the action in terms of the free theory and the interacting theory

\begin{align*}
S &= S_0 + S_{int} \\
S &= \left( \int d^4 x \, (\bar{\psi} ( i \slashed{\partial} - m_f ) \psi - \frac{1}{4} F^{\mu\nu} F_{\mu\nu} + \frac{1}{2} m_b^2 A_\mu A^\mu \right) \\ 
&+ \left( -ie \int d^4 x \, \bar{\psi} A_\mu \gamma^\mu \psi  \right).
\end{align*}

\noindent Then the quantized 2-point correlation function for massive QED with the Taylor expansion is an infinite series of $n$-point correlation functions for the Grassmann-valued Gaussian path integrals

\begin{equation}
\bra{0} \mathcal{T} [ \hat{\psi} (x) \hat{\bar{\psi}} (y) ] \ket{0} = \lim_{T \rightarrow \infty (1-i \epsilon)} \frac{\int \mathcal{D} \psi \mathcal{D} \bar{\psi} \mathcal{D} A \,\, e^{i S_0} \psi (x) \bar{\psi} (y) ( 1 - i e \int d^4 x \, \bar{\psi} A_\mu \gamma^\mu \psi + \dots ) }{\int \mathcal{D} \psi \mathcal{D} \bar{\psi} \mathcal{D} A \,\, e^{i(S_0 + S_{int})}}.
\end{equation}

\noindent The Feynman rules for massive QED, which can be deduced via patterns from the Taylor series, are

\begin{enumerate}
\item Draw a straight line from $a$ to $b$ with momenta $p$ for each fermion and associate $\left( \frac{i}{\slashed{p} - m_f + i \epsilon} \right)_{ab}$
\item Draw squiggly line from $\alpha$ to $\beta$ with momenta $q$ for each boson and associate $\left( \frac{-i}{k^2 - m_b^2 + i \epsilon} \right) \delta_{\alpha \beta}$
\item To each vertex associate $i e \gamma^\mu$
\item Enforce momentum conservation at vertices: $(2\pi)^4 \delta^{(4)} (\Sigma_{in} p - \Sigma_{out} q)$
\item Integrate over undetermined momenta
\item Amputate external lines
\item Incoming fermions  $\eta^a (p)$ and outgoing fermions $\bar{\eta}^b (p)$
\item $(-1)$ for each closed fermion loop.
\end{enumerate}

\noindent This prescription creates infinities, and we will visit renormalization to tame these infinities in the following lectures.

%\clearpage

\end{document}

